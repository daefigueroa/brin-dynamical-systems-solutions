\subsection*{Exercise 2.8.3.}

\begin{exercise}{stm:ex2.8.3-283statement}
Prove the following generalization of Proposition 2.1.2. If a commutative group $G$ acts by homeomorphisms on a compact metric space $X$, then there is a non-empty, closed $G$-invariant subset $X'$ on which $G$ acts minimally.
\end{exercise}

\subsection*{Proof {\color{blue}+ reasoning}:}

\begin{explanation}{stm:ex2.8.3-1}
Let's try to adapt the proof of Proposition 2.1.2 to the more general case.
\end{explanation}

\begin{explanation}{stm:ex2.8.3-2}
There are four theorems stated in section 2.8, but they don't seem applicable in this exercise.
\end{explanation}

\begin{statement}{stm:ex2.8.3-3}
Let $\mathcal{C}$ be the collection of non-empty, closed $G$-invariant subsets of $X$, with the partial ordering given by inclusion.
\end{statement}

\begin{statement}{stm:ex2.8.3-4}
Since $X \in \mathcal{C}$, $\mathcal{C}$ is not empty.
\end{statement}

\begin{statement}{stm:ex2.8.3-5}
Suppose $\mathcal{K} \subseteq \mathcal{C}$ is a totally ordered subset. Then, any finite intersection of elements of $\mathcal{K}$ is nonempty, so by the finite intersection property for compact sets, $\bigcap_{K \in \mathcal{K}} K \ne \emptyset$. Thus, by Zorn's lemma, $\mathcal{C}$ contains a minimal element $M$.
\end{statement}

\begin{explanation}{stm:ex2.8.3-6}
So far, we have followed the proof of 2.1.2 almost exactly. How can we conclude that $G$ acts minimally on $M$?
\end{explanation}

\begin{explanation}{stm:ex2.8.3-7}
Intuitively speaking, we have very little constructive information about $M$, so proof by contradiction seems like a good strategy.
\end{explanation}

\begin{statement}{stm:ex2.8.3-8}
Suppose that $G$ does not act minimally on $M$.
\end{statement}

\begin{statement}{stm:ex2.8.3-9}
Then, there exists a point $b \in M$ and a nonempty open set $C \subseteq M$ such that $Gb \cap C = \emptyset$.
\end{statement}

\begin{explanation}{stm:ex2.8.3-10}
To conclude the proof, we should find a contradiction, given (\ref{stm:ex2.8.3-9}) and (\ref{stm:ex2.8.3-5}). We want to find some set $Y \subseteq M$ that is closed, nonempty, and $G$-invariant. Let's start with some possible natural choices for $Y$, and work from there.
\end{explanation}

\begin{explanation}{stm:ex2.8.3-11}
Note $C$ is not closed, $\text{cl}(C)$ is not necessarily $G$-invariant, and $M \setminus C$ is also not necessarily $G$-invariant. On the other hand, $M \setminus GC$ seems like a good candidate.
\end{explanation}

\begin{statement}{stm:ex2.8.3-12}
Since $GC = \bigcup_{g \in G} gC$, and each $g \in G$ is a homeomorphism, $GC$ is open.
\end{statement}

\begin{statement}{stm:ex2.8.3-13}
So, since $M$ is closed, $M \setminus GC$ is closed.
\end{statement}

\begin{statement}{stm:ex2.8.3-14}
Since $b \in M$ and $Gb \cap C = \emptyset$, $b \in M \setminus GC$, so $M \setminus GC$ is nonempty.
\end{statement}

\begin{statement}{stm:ex2.8.3-15}
If $m \in M \setminus GC$ and $g \in G$, then $m \ne g^{-1}c$, so $gm \ne C$, so since $M$ is $G$-invariant, $gm \in M \setminus GC$, so $M \setminus GC$ is $G$-invariant.
\end{statement}

\begin{statement}{stm:ex2.8.3-16}
By (\ref{stm:ex2.8.3-13})–(\ref{stm:ex2.8.3-15}), $M \setminus GC$ is a closed, nonempty, $G$-invariant proper subset of $M$, which contradicts (\ref{stm:ex2.8.3-5}), so (\ref{stm:ex2.8.3-8}) is false. Hence, $G$ acts minimally on $M$.
\end{statement}