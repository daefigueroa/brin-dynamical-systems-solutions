\sectionlabel{Exercise 2.2.3.}

\begin{stm}{stm:223setup1}[exercise]
Is the product of two topologically transitive systems topologically transitive?
\end{stm}

\begin{stm}{stm:223setup2}[exercise]
Is a factor of a topologically transitive system topologically transitive?
\end{stm}

\sectionlabel{Proof {\color{blue}+ reasoning}:}

\begin{stm}{stm:1}[reasoning]
I think the statement is false. Let's construct a counterexample, by starting with the simplest non-trivial product of topologically transitive systems, and iterating from there.
\end{stm}

\begin{stm}{stm:2}[proof]
Let $R_\alpha$ be the circle translation, where $\alpha$ is irrational.
\end{stm}

\begin{stm}{stm:3}[proof]
It is known that $R_\alpha$ is topologically transitive.
\end{stm}

\begin{stm}{stm:4}[reasoning]
$R_\alpha \times R_\alpha$ already seems to form a counterexample, since $R_\alpha \times R_\alpha$ preserves the distance between the components of points in $S^1 \times S^1$.
\end{stm}

\begin{stm}{stm:5}[proof]
Let $(a,b) \in S^1 \times S^1$. 
\end{stm}

\begin{stm}{stm:5b}[proof]
If $a \geq b$, then the orbit of $(a,b)$ under $R_\alpha \times R_\alpha$ is contained in $l_1 \cup l_2$ where
\begin{align*}
l_1 &= \left\{ t(a-b,0) + (1 - t)(1, b-a+1) : t \in [0,1] \right\}, \\
l_2 &= \left\{ t(1, b-a+1) + (1 - t)(1, b-a+1) : t \in [0,1] \right\}.
\end{align*}
\end{stm}

\begin{stm}{stm:6}[proof]
If $b \geq a$, then the same holds with
\begin{align*}
l_1 &= \left\{ t(0, a-b+1) + (1 - t)(b - a, 1) : t \in [0,1] \right\}, \\
l_2 &= \left\{ t(b-a, 0) + (1 - t)(1, a-b+1) : t \in [0,1] \right\}.
\end{align*}
\end{stm}

\begin{stm}{stm:7}[proof]
In both cases, $l_1$ and $l_2$ are lines contained in $[0,1) \times [0,1)$. Since these lines are clearly not dense in $S^1 \times S^1$, the forward orbit of $(a,b)$ is not dense in $S^1 \times S^1$. Hence, $R_\alpha \times R_\alpha$ is not topologically transitive.
\end{stm}

\begin{stm}{stm:insert9}[proof]
Suppose $f: X \to X$ and $g: Y \to Y$ are topological dynamical systems, that $\pi$ is a topological semiconjugacy from $f$ to $g$, and that $f$ is topologically transitive with point $x \in X$ with dense forward orbit.
\end{stm}

\begin{stm}{stm:10}[reasoning]
We want to show that the forward orbit of $\pi(x)$ is dense.
\end{stm}

\begin{stm}{stm:11}[proof]
Let $U \subseteq Y$ be open. Since $\pi$ is continuous, $\pi^{-1}(U)$ is open, so by \ref{stm:insert9}, there exists a $k \in \mathbb{N}$ such that $f^k(x) \in \pi^{-1}(U)$.
\end{stm}

\begin{stm}{stm:12}[proof]
By \ref{stm:insert9}, $\pi \circ f^k(x) = g^k(\pi(x))$.
\end{stm}

\begin{stm}{stm:13}[proof]
By \ref{stm:11} and \ref{stm:12}, $g^k(\pi(x)) \in U$, so $\pi(x)$ is dense, hence a factor of a topologically transitive system is topologically transitive.
\end{stm}
