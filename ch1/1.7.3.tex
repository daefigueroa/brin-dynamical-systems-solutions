\subsection*{Exercise 1.7.3.}

\begin{exercise}{stm:ex1.7.3-setup173}
Show that the eigenvalues of a two-dimensional hyperbolic toral automorphism are irrational (so the stable and unstable manifolds are dense by exercise 1.11.1).
\end{exercise}

\subsection*{Proof {\color{blue} + reasoning}:}

\begin{explanation}{stm:ex1.7.3-q3}
Are there hyperbolic toral automorphisms that aren't represented by a matrix? No, not in this context.
\end{explanation}

\begin{statement}{stm:ex1.7.3-4}
Let $A$ be a $2 \times 2$ integer matrix such that $\det(A) = 1$ and such that for all eigenvalues $\lambda$ of $A$, $|\lambda| \neq 1$.
\end{statement}

\begin{statement}{stm:ex1.7.3-5}
Let $\lambda$ be an eigenvalue of $A$.
\end{statement}

\begin{explanation}{stm:ex1.7.3-6}
Let's try the following: first, relate the eigenvalues to the determinant. Then, conclude from the first step and the given assumptions that the eigenvalues are irrational.
\end{explanation}

\begin{statement}{stm:ex1.7.3-8}
Note that $\det(\lambda I - A) = 0$.
\end{statement}

\begin{statement}{stm:ex1.7.3-8a}
Denote $A = \begin{bmatrix} a_{11} & a_{12} \\ a_{21} & a_{22} \end{bmatrix}$.
\end{statement}

\begin{statement}{stm:ex1.7.3-9}
$(\lambda - a_{11})(\lambda - a_{22}) - a_{12}a_{21} = 0$
\end{statement}

\begin{statement}{stm:ex1.7.3-10}
$\lambda^2 - \lambda a_{11} - \lambda a_{22} + a_{11}a_{22} - a_{12}a_{21} = 0$
\end{statement}

\begin{statement}{stm:ex1.7.3-12}
By assumption, $\det(A) = 1$, so $a_{11}a_{22} - a_{12}a_{21} = 1$
\end{statement}

\begin{statement}{stm:ex1.7.3-14}
Substituting $1 = a_{11}a_{22} - a_{12}a_{21}$ in (\ref{stm:ex1.7.3-10}) gives
$$\lambda^2 - \lambda(a_{11} + a_{22}) + 1 = 0$$
\end{statement}

\begin{explanation}{stm:ex1.7.3-14b}
That was the first step. How can we conclude? What given assumption have I not used yet?
\end{explanation}

\begin{explanation}{stm:ex1.7.3-r16}
I haven't used that $|\lambda| \ne 1$. From (\ref{stm:ex1.7.3-14}) it may be possible to conclude that $\lambda$ is equal to $1$ or irrational, which together with (\ref{stm:ex1.7.3-r16}) gives that $\lambda$ is irrational. We can factorize the left-hand side of (\ref{stm:ex1.7.3-14}). However, (\ref{stm:ex1.7.3-14}) could have complex solutions for certain values of $a_{11} + a_{22}$. If the discriminant of (\ref{stm:ex1.7.3-14}) is greater than $0$, (\ref{stm:ex1.7.3-14}) only has real roots. 
\end{explanation}

\begin{explanation}{stm:ex1.7.3-27}
Let's try a different approach. Equation (\ref{stm:ex1.7.3-14}) is quadratic in $\lambda$, so all solutions are given by the quadratic formula.
\end{explanation}

\begin{statement}{stm:ex1.7.3-15}
By the quadratic formula, $\lambda = \frac{1}{2}(a_{11} + a_{22} + ((a_{11} + a_{22})^2 - 4)^{1/2})$ or $\lambda = \frac{1}{2}(a_{11} + a_{22} - ((a_{11} + a_{22})^2 - 4)^{1/2})$
\end{statement}

\begin{statement}{stm:ex1.7.3-16}
From (\ref{stm:ex1.7.3-15}) we see that $\lambda \in \mathbb{R} \setminus \mathbb{Q}$ if and only if $((a_{11} + a_{22})^2 - 4)^{1/2} \in \mathbb{R} \setminus \mathbb{Q}$
\end{statement}

\begin{statement}{stm:ex1.7.3-17}
By definition of $A$, $(a_{11} + a_{22}) \in \mathbb{N}$
\end{statement}

\begin{explanation}{stm:ex1.7.3-insert18a}
Let's show that $\lambda$ cannot be in $\mathbb{C}$. My guess is that if $\lambda$ were in $\mathbb{C}$, its magnitude would be equal to $1$.
\end{explanation}

\begin{statement}{stm:ex1.7.3-a18}
Suppose $((a_{11} + a_{22})^2 - 4)^{1/2} \in \mathbb{C}$. Then $|\lambda|^2 = \frac{1}{4}((a_{11} + a_{22})^2 + 4 - (a_{11} + a_{22})^2) = 1$
\end{statement}

\begin{statement}{stm:ex1.7.3-20}
(\ref{stm:ex1.7.3-a18}) contradicts (\ref{stm:ex1.7.3-4}), hence $((a_{11} + a_{22})^2 - 4)^{1/2} \in \mathbb{R}$
\end{statement}

\begin{statement}{stm:ex1.7.3-21}
From (\ref{stm:ex1.7.3-20}) and (\ref{stm:ex1.7.3-17}), $a_{11} + a_{22} \geq 3$
\end{statement}

\begin{explanation}{stm:ex1.7.3-22}
We just need to show that $\lambda \notin \mathbb{Q}$. My intuition is that $((a_{11} + a_{22})^2 - 4)^{1/2}$ is always irrational, because it seems $\forall n \in \mathbb{N}$, $n^{1/2}$ is rational only if $n$ is a square number, and subtracting $4$ makes it no longer square, given $\sqrt{n} \ge 3$. Let's prove the first part.
\end{explanation}

\begin{conj}\label{stm:ex1.7.3-c23}
$\{n \in \mathbb{N} : \sqrt{n} \in \mathbb{Q} \setminus \mathbb{N} \} = \emptyset$
\end{conj}

\begin{proof}
Suppose for contradiction that $\sqrt{n} \in \mathbb{Q} \setminus \mathbb{N}$. Then $\sqrt{n} = \frac{p}{q}$, where $p$ and $q$ are natural numbers. Since $\sqrt{n} \notin \mathbb{N}$, $q$ does not divide $p$, so $q^2$ does not divide $p^2$, but $n = \frac{p^2}{q^2}$, hence $n q^2 = p^2$ which is a contradiction.
\end{proof}

\begin{explanation}{stm:ex1.7.3-insert33a}
Now let's prove the second part, that subtracting 4 makes the number no longer square. Intuitively, this seems true because the distance between consecutive squares will eventually be greater than any fixed number, like 4, so by subtracting a fixed number from large enough squares, we end up in between squares.
\end{explanation}

\begin{conj}\label{stm:ex1.7.3-c34}
$\forall n \in \mathbb{N}, n \ge 3$ implies $(n^2 - 4)^{1/2} \notin \mathbb{N}$
\end{conj}

\begin{proof}
Suppose $(n^2 - 4)^{1/2} = k$ where $k \in \mathbb{N}$. Then $n^2 - 4 = k^2$, so $n^2 - k^2 = 4$. Clearly, $n > k$. Then $n^2 - k^2 \ge n^2 - (n - 1)^2 = n^2 - n^2 + 2n - 1 = 2n - 1$. So $n^2 - k^2 > 4$, which is a contradiction.
\end{proof}

\begin{statement}{stm:ex1.7.3-43}
Now we can conclude. By (\ref{stm:ex1.7.3-21}) and (\ref{stm:ex1.7.3-c34}), $((a_{11} + a_{22})^2 - 4)^{1/2} \notin \mathbb{N}$
\end{statement}

\begin{statement}{stm:ex1.7.3-44}
So, by (\ref{stm:ex1.7.3-43}), (\ref{stm:ex1.7.3-20}) and (\ref{stm:ex1.7.3-c23}), $((a_{11} + a_{22})^2 - 4)^{1/2}$ is irrational
\end{statement}

\begin{statement}{stm:ex1.7.3-45}
So, by (\ref{stm:ex1.7.3-44}) and (\ref{stm:ex1.7.3-15}), $\lambda$ is irrational
\end{statement}

\begin{explanation}{stm:ex1.7.3-insert45a}
We still need to show that the (un)stable manifolds are dense by exercise 1.11.1.
\end{explanation}

\begin{statement}{stm:ex1.7.3-46}
Let $x \in \mathbb{T}^2$. Without loss of generality, $\lambda > 1 > \lambda^{-1}$, where $\lambda$ and $\lambda^{-1}$ are the eigenvalues of $A$. The stable manifold $W^u(x)$ is the line through $x$ parallel to $v$ where $v$ is the eigenvector corresponding to $\lambda$.
\end{statement}

\begin{explanation}{stm:ex1.7.3-insert47a}
I think the intended proof is to show that the slope of a line parallel to $v$ is irrational, hence the flow defined in section 1.11 has dense orbits, from which it follows $W^u(x)$ is dense.
\end{explanation}

\begin{statement}{stm:ex1.7.3-49}
Denote $v = \begin{pmatrix} v_1 \\ v_2 \end{pmatrix}$. A line parallel to $v$ has slope equal to $\frac{v_2}{v_1}$
\end{statement}

\begin{explanation}{stm:ex1.7.3-insert50a}
My guess is that the slope is irrational because $\lambda$ is irrational and because $v$ is an eigenvector of an integer-valued matrix. Let's write out the defining equation for eigenvectors.
\end{explanation}

\begin{statement}{stm:ex1.7.3-52}
$A v = \lambda v$
\end{statement}

\begin{statement}{stm:ex1.7.3-53}
$\begin{bmatrix} a_{11}v_1 + a_{12}v_2 \\ a_{21}v_1 + a_{22}v_2 \end{bmatrix} = \begin{bmatrix} \lambda v_1 \\ \lambda v_2 \end{bmatrix}$
\end{statement}

\begin{statement}{stm:ex1.7.3-55}
$a_{11}v_1 + a_{12}v_2 = \lambda v_1$
\end{statement}

\begin{statement}{stm:ex1.7.3-56}
$a_{21}v_1 + a_{22}v_2 = \lambda v_2$
\end{statement}

\begin{statement}{stm:ex1.7.3-57}
$v_2(\lambda - a_{22}) = a_{21}v_1$
\end{statement}

\begin{statement}{stm:ex1.7.3-58}
$v_1(\lambda - a_{11}) = a_{12}v_2$
\end{statement}

\begin{explanation}{stm:ex1.7.3-59b}
If $a_{21} = 0$, then $v_2 v_1^{-1} = 0$. Why is this not possible? Probably because $v$ is nonzero, by assumption.
\end{explanation}

\begin{statement}{stm:ex1.7.3-59c}
Since $v$ is an eigenvector, $v_1 \ne 0$ or $v_2 \ne 0$
\end{statement}

\begin{statement}{stm:ex1.7.3-59e}
By (\ref{stm:ex1.7.3-45}), $(\lambda - a_{22})$ and $(\lambda - a_{11})$ are irrational.
\end{statement}

\begin{statement}{stm:ex1.7.3-59f}
If $v_1 \ne 0$, then by (\ref{stm:ex1.7.3-59e}) and (\ref{stm:ex1.7.3-58}), $v_2 \ne 0$
\end{statement}

\begin{statement}{stm:ex1.7.3-59g}
If $v_2 \ne 0$, then by (\ref{stm:ex1.7.3-59e}) and (\ref{stm:ex1.7.3-57}), $v_1 \ne 0$
\end{statement}

\begin{statement}{stm:ex1.7.3-59h}
So, $v_1 \ne 0$ and $v_2 \ne 0$
\end{statement}

\begin{statement}{stm:ex1.7.3-59i}
By (\ref{stm:ex1.7.3-59h}) and (\ref{stm:ex1.7.3-57}), $v_2 v_1^{-1} = (\lambda - a_{22})^{-1} a_{21} \ne 0$
\end{statement}

\begin{statement}{stm:ex1.7.3-59j}
By (\ref{stm:ex1.7.3-59i}) and (\ref{stm:ex1.7.3-59e}), $v_2 v_1^{-1}$ is irrational
\end{statement}

\begin{statement}{stm:ex1.7.3-62}
Denote $x = (x_1, x_2)$, let $t \in \mathbb{R}^+$
\end{statement}

\begin{statement}{stm:ex1.7.3-63}
Let $\phi_{\frac{v_1}{v_2}}^t(x) := (x_1 + \frac{v_1}{v_2} t, x_2 + t) \mod 1$
\end{statement}

\begin{statement}{stm:ex1.7.3-64}
Then $\bigcup_{t \in \mathbb{R}^+} \phi_{\frac{v_1}{v_2}}^t(x) \subseteq W^u(x)$
\end{statement}

\begin{statement}{stm:ex1.7.3-65}
By exercise 1.11.1, the orbit of $\phi_{\frac{v_1}{v_2}}$ is dense, so $W^u(x)$ is dense. For the stable manifold, the proof is similar.
\end{statement}
