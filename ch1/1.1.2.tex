\subsection*{Exercise 1.1.2}

\begin{exercise}{stm:112eexercise}
Suppose $(X,f)$ is a factor of $(Y,g)$ by a semi-conjugacy $\pi\colon Y \to X$.
\end{exercise}

\begin{exercise}{stm:112exsetup}
Show that if $y \in Y$ is a periodic point, then $\pi(y) \in X$ is periodic.
\end{exercise}

\begin{exercise}{stm:112exsetup2}
Give an example to show that the preimage of a periodic point does not necessarily contain a periodic point.
\end{exercise}

\subsection*{Proof {\color{blue} + reasoning}:}

\begin{explanation}{reas1}
Let's start with (\ref{stm:112exsetup}).
\end{explanation}

\begin{statement}{stm:112exverify}
Let $y \in Y$ be periodic.
\end{statement}

\begin{explanation}{stm:112exwrite}
We write the definitions and see if the correct answer follows.
\end{explanation}

\begin{statement}{stm:112excalc}
$$
f(\pi(y)) \;=\; \pi\bigl(g(y)\bigr) \;=\; \pi(y).
$$
\end{statement}

\begin{statement}{stm:112excalc-conclusion}
So, (\ref{stm:112exsetup}) follows from (\ref{stm:112excalc}).
\end{statement}

\begin{explanation}{stm:112excounterexample-intro}
Now, we want to construct a counterexample. Let's think of any semiconjugacy and iterate from there. An obvious example is a projection.
\end{explanation}

\begin{statement}{stm:112exAB-sets}
Let $A$ and $B$ be sets, and $\pi_A\colon A \times B \to A$ the projection.
\end{statement}

\begin{statement}{stm:112exdiagram}
For all $\alpha$ and $\beta$, the following diagram commutes:
\[
\begin{array}{c@{\quad}c@{\quad}c}
A \times B & \xrightarrow{\;\alpha \times \beta\;} & A \times B\\[6pt]
\downarrow{\pi_A} && \downarrow{\pi_A}\\[4pt]
A & \xrightarrow{\;\alpha\;} & A
\end{array}
\]
\end{statement}

\begin{explanation}{stm:112exintuition}
Intuitively, by taking projections, we 'forget' about the effect of $\beta$. So, we can simply choose $\beta$ such that all points in $A \times B$ are non-periodic with respect to $\beta$, while choosing $\alpha$ so that all points in $A$ are periodic with respect to $\alpha$. 
\end{explanation}

\begin{explanation}{stm:112exfinal-example}
Let $A$ be any set, $B = [0,1]$, $\alpha = \mathrm{id}_A$ and $\beta: (x \mapsto \frac{1}{2}x)$
\end{explanation}

\begin{explanation}{reas2}
We note $0$ is still a periodic point in the example above. Let's modify it.
\end{explanation}

\begin{statement}{stm:112exfinal-example2}
    Let $A$ be any nonempty set, $B = (0,1]$, $\alpha = \mathrm{id}_A$ and $\beta: (x \mapsto \frac{1}{2}x)$
\end{statement}
    
\begin{statement}{stm:112excounterexample}
Clearly, $\beta$ has no periodic points, so $\alpha \times \beta$ has no periodic points, but all points in $A$ are periodic with respect to $\alpha$. 
\end{statement}

\begin{statement}{}
By (\ref{stm:112excounterexample}), (\ref{stm:112exsetup2}) follows.
\end{statement}