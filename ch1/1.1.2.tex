\subsection*{Exercise 1.1.2}

\begin{exercise}{stm:112eexercise}
Suppose $(X,f)$ is a factor of $(Y,g)$ by a semi-conjugacy $\pi\colon Y \to X$.
\end{exercise}

\begin{exercise}{stm:112exsetup}
Show that if $y \in Y$ is a periodic point, then $\pi(y) \in X$ is periodic.
\end{exercise}

\begin{exercise}{stm:112exsetup2}
Give an example to show that the preimage of a periodic point does not necessarily contain a periodic point.
\end{exercise}

\subsection*{Proof {\color{blue} + reasoning}:}

\begin{explanation}{reas1}
Let's start with (\ref{stm:112exsetup}).
\end{explanation}

\begin{statement}{stm:112exverify}
Let $y \in Y$ be periodic. Then there exists $n \in \mathbb{N}$ such that $g^n(y) = y$.
\end{statement}

\begin{statement}{stm:112excalc}
Since $f$ is a factor of $g$,
$$
f^n(\pi(y)) = \pi(g^n(y)) = \pi(y).
$$
\end{statement}

\begin{statement}{stm:112excalc-conclusion}
So, $\pi(y)$ is periodic.
\end{statement}

\begin{explanation}{stm:112excounterexample-intro}
Now, we want to construct a counterexample. Let's think of any semiconjugacy and iterate from there. An obvious example is a projection.
\end{explanation}

\begin{statement}{stm:112exAB-sets}
Let $A$ and $B$ be sets, and $\pi_A\colon A \times B \to A$ the projection.
\end{statement}

\begin{statement}{stm:112exdiagram}
For all $\alpha$ and $\beta$, the following diagram commutes:
\[
\begin{array}{c@{\quad}c@{\quad}c}
A \times B & \xrightarrow{\;\alpha \times \beta\;} & A \times B\\[6pt]
\downarrow{\pi_A} && \downarrow{\pi_A}\\[4pt]
A & \xrightarrow{\;\alpha\;} & A
\end{array}
\]
\end{statement}

\begin{explanation}{stm:112exintuition}
Intuitively, by taking projections, we obtain a semiconjugacy regardless of the choice of $\beta$, and non-periodic points are invariant under taking products. So, we can choose $\beta: B \rightarrow B$ such that all points in $B$ (and hence in $A \times B$) are non-periodic, while choosing $\alpha: A \rightarrow A$ such that all points in $A$ are periodic.
\end{explanation}

\begin{explanation}{stm:112exfinal-example}
Let $A$ be any set, $B = [0,1]$, $\alpha = \mathrm{id}_A$ and $\beta: (x \mapsto \frac{1}{2}x)$
\end{explanation}

\begin{explanation}{reas2}
We note $0$ is still a periodic point in the example above. Let's modify it.
\end{explanation}

\begin{statement}{stm:112exfinal-example2}
    Let $A$ be any nonempty set, $B = (0,1]$, $\alpha = \mathrm{id}_A$ and $\beta: (x \mapsto \frac{1}{2}x)$
\end{statement}
    
\begin{statement}{stm:112excounterexample}
Clearly, $\beta$ has no periodic points, so $\alpha \times \beta$ has no periodic points, but all points in $A$ are periodic with respect to $\alpha$. 
\end{statement}