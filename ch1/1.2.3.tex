\subsection*{Exercise 1.2.3}

\begin{exercise}{stm:setup}
Let $G$ be a topological group.  
\end{exercise}

\begin{exercise}{stm:x1}
Prove that for each $g \in G$, the closure $H(g)$ of the set $\{g^n\}_{n=-\infty}^\infty$ is a commutative subgroup of $G$.
\end{exercise}

\begin{exercise}{stm:x2}
Thus, if $G$ has a minimal left translation, then $G$ is abelian.
\end{exercise}

\subsection*{Proof {\color{blue} + reasoning}:}

\begin{explanation}{stm:d3}
First, let's show the closure of $\{g^n\}_{n=-\infty}^\infty$ is a subgroup of $G$, starting with showing closure under the group operation.
\end{explanation}

\begin{statement}{stm:d3def}
Define ${\langle g \rangle} := \{g^n\}_{n=-\infty}^\infty$.
\end{statement}

\begin{statement}{stm:a4}
Let $g \in G$. Let $a, b \in \mathrm{cl}({\langle g \rangle})$.
\end{statement}

\begin{explanation}{stm:r4}
What do I know about closures? The closure of $A$ is the set $X$ of points such that any neighborhood of $x \in X$ contains a point in $A$.
\end{explanation}

\begin{explanation}{stm:r5}
My intuition is that the required proof will resemble the one I would follow if $G$ were a metric space. In a metric space, if $a$ and $b$ are limits of $g^{k_n}$ and $g^{l_n}$ we should have $g^{k_n} g^{l_n} \to ab$. Here we do not have a metric, so there is no notion of convergent sequences, but instead of neighborhoods: $x$ is a limit point of $A$ if every neighborhood of $x$ contains a point in $A$ other than $x$ itself.
\end{explanation}

\begin{explanation}{stm:d6}
Intuitively, since $a$ and $b$ are in ${\langle g \rangle}$ or limit points of ${\langle g \rangle}$, the product of the two points that ‘witness’ this property should be the point that witnesses $ab$ being a limit point.
\end{explanation}

\begin{explanation}{stm:l7}
Let $C$ be a neighborhood of $ab$, and $U \subseteq C$ an open set containing $ab$.
\end{explanation}

\begin{explanation}{stm:8}
$a^{-1}U$ and $Ub^{-1}$ are open.
\end{explanation}

\begin{explanation}{stm:9}
$b \in a^{-1}U$ and $a \in Ub^{-1}$. Since $a$ and $b$ are limit points of ${\langle g \rangle}$, $\exists k, m \in \mathbb{Z}$ such that $g^m \in a^{-1}U$ and $g^k \in Ub^{-1}$.
\end{explanation}

\begin{explanation}{stm:10}
$g^k g^m$ should be in $U$, but I can't show why. What tools can I give myself to help prove $g^k g^m \in U$?
\end{explanation}

\begin{explanation}{stm:r11}
Well, $g^m$ and $g^k$ are homeomorphisms, so $g^k g^m \in g^k a^{-1}U$, $g^k g^m \in Ub^{-1} g^m$, and $(g^k a^{-1}U) \cup (Ub^{-1} g^m)$ is open.
\end{explanation}

\begin{explanation}{stm:r11b}
Now I am stuck.
\end{explanation}

\begin{explanation}{stm:m12}
What given assumptions have I not used? 
\end{explanation}

\begin{explanation}{stm:m12b}
I have not used the fact that the group operation $G \times G \to G$ is continuous. I only used that, $\forall g \in G$, left and right multiplication by $g$ is a continuous function $G \to G$, which seems to be a weaker statement. Using the ‘joint’ continuity should work.
\end{explanation}

\begin{statement}{stm:13}
Since the group multiplication $\alpha: G \times G \to G$ is continuous, $\alpha^{-1}(U)$ is open in $G \times G$. Since $(a,b) \in \alpha^{-1}(U)$, and since sets of the form $A \times B$, where $A$ and $B$ are open, form a basis for the topology on $G \times G$, there exist open $V$ and $W$ such that $a \in V$, $b \in W$, and such that $V \times W \subseteq \alpha^{-1}(U)$.
\end{statement}

\begin{statement}{stm:14}
Since $a,b \in \mathrm{cl}({\langle g \rangle})$ there exist $g^\ell \in V$ and $g^p \in W$. By (\ref{stm:13}), $g^\ell g^p \in U$, hence $g^{\ell+p} \in U$, so $ab \in \mathrm{cl}({\langle g \rangle}) = H(g)$.
\end{statement}

\begin{statement}{stm:15}
By (\ref{stm:14}), $H(g)$ is closed under taking products.
\end{statement}

\begin{explanation}{stm:16}
Now we need to show that $H(g)$ has inverses, by showing $a^{-1} \in H(g)$.
\end{explanation}

\begin{statement}{stm:17}
Let $C$ be a neighborhood of $a^{-1}$ and $U \subseteq C$ an open set such that $a^{-1} \in U$. Since the inverse is continuous, $U' := \{x \in G : x^{-1} \in U\}$ is open, and it contains $a$.
\end{statement}

\begin{statement}{stm:18}
Since $a \in H(g)$, there exists $g^q \in U'$, where $q \in \mathbb{Z}$.
\end{statement}

\begin{statement}{stm:20}
By (\ref{stm:18}), $g^{-q} = (g^q)^{-1} \in U$, so $a^{-1} \in H(g)$.
\end{statement}

\begin{statement}{stm:21}
By (\ref{stm:20}), $H(g)$ is closed under taking inverses.
\end{statement}

\begin{explanation}{stm:q23}
Now to prove that $H(g)$ is commutative.
\end{explanation}

\begin{explanation}{stm:q24}
We need to show that $ab = ba$. If $G$ were a metric space, the proof would follow from the fact that the limits of convergent sequences are unique. Is there something like uniqueness of limit points in a general topological space? The answer seems to be no, only when adding separation properties.
\end{explanation}

\begin{explanation}{stm:q25}
Let’s take a few steps back and try again. Note, the product in $H(g)$ is just the restriction of the one in $G$, so if $ab \ne ba$ in $G$, $ab \ne ba$ in $H(g)$. So, the only way in which $H(g)$ can be commutative is if it excludes at least all non-commutative elements in $G$.
\end{explanation}

\begin{explanation}{stm:r25}
So, $H(g)$ must be a proper subgroup if $G$ is not abelian. Considering (\ref{stm:q25}), I think we should try to prove the contrapositive instead, i.e. prove if two elements of $G$ are not commutative, then at least one of them is not in $H(g)$.
\end{explanation}

\begin{statement}{stm:l26}
  Let $c, d \in G$ with $cd \ne dc$.
\end{statement}

\begin{explanation}{stm:q27}
Why is $(c,d) \notin H(g) \times H(g)$? I am stuck here.
\end{explanation}

\begin{explanation}{stm:m28}
Why has my best attempt not worked? To show (\ref{stm:q27}), we need to show that there exists a neighborhood of $(c,d)$ containing no element of ${\langle g \rangle}$, but I can't find any obvious neighborhood. There is no given neighborhood from the definitions. I think the exercise is not correct without adding a separation property, so let's add it ourselves.
\end{explanation}

\begin{statement}{stm:29}
Suppose that $G$ is Hausdorff. 
\end{statement}

\begin{statement}{stm:30}
By (\ref{stm:29}) and (\ref{stm:l26}), there exist open neighborhoods $U$ of $cd$ and $U'$ of $dc$ such that $U \cap U' = \emptyset$.
\end{statement}

\begin{statement}{stm:a32}
Suppose $c,d \in H(g)$.
\end{statement}

\begin{statement}{stm:33}
Similarly to (\ref{stm:13}), $(c,d) \in \alpha^{-1}(U)$ and $(d,c) \in \alpha^{-1}(U')$.
\end{statement}

\begin{statement}{stm:34}
So there are open sets $V,V',W,W'$ such that $(c,d) \in V \times W \subseteq \alpha^{-1}(U)$ and $(d,c) \in V' \times W' \subseteq \alpha^{-1}(U')$.
\end{statement}

\begin{statement}{stm:35}
From (\ref{stm:34}), $c \in V \cap V'$ and $d \in W \cap W'$, and $V \cap V'$ and $W \cap W'$ are open.
\end{statement}

\begin{statement}{stm:36}
So, by (\ref{stm:a32}), there exist $s,t \in \mathbb{Z}$ such that $g^s \in V \cap V'$ and $g^t \in W \cap W'$.
\end{statement}

\begin{statement}{stm:37}
By (\ref{stm:36}), $(g^s, g^t) \in V \times W$ and $(g^t, g^s) \in W' \times V'$.
\end{statement}

\begin{statement}{stm:38}
By (\ref{stm:37}) and (\ref{stm:34}), $g^s g^t \in U$ and $g^t g^s \in U'$, so $g^{t+s} \in U \cap U'$.
\end{statement}

\begin{statement}{stm:39}
(\ref{stm:38}) contradicts (\ref{stm:29}), so (\ref{stm:a32}) is false, hence $c \notin H(g)$ or $d \notin H(g)$, so $H(g)$ is commutative.
\end{statement}

\begin{statement}{stm:40}
By (\ref{stm:39}), (\ref{stm:15}) and (\ref{stm:21}), $H(g)$ is a commutative subgroup of $G$.
\end{statement}

\begin{explanation}{40b}
We still need to prove that if $G$ has a minimal left translation, then $G$ is Abelian.
\end{explanation}

\begin{statement}{stm:a41}
Suppose that $G$ has a minimal left translation $L_h : G \to G$ where $h \in G$.
\end{statement}

\begin{statement}{stm:a41b}
By (\ref{stm:40}), $H(h)$ is a commutative subgroup of $G$.
\end{statement}

\begin{statement}{stm:42}
By definition, $L_h$ has no proper closed non-empty invariant subsets.
\end{statement}

\begin{statement}{stm:43}
$H(h)$ is a closed non-empty subset of $G$.
\end{statement}

\begin{explanation}{stm:44}
Is $H(h)$ invariant with respect to $L_h$?
\end{explanation}

\begin{statement}{stm:45}
Let $a \in H(h)$. Let $C$ be a neighborhood of $ha$ and $U$ open with $ha \in U \subseteq C$. $a \in h^{-1}U$, and $h^{-1}U$ is open, so $\exists q \in \mathbb{Z}$ such that $h^q \in h^{-1}U$.
\end{statement}

\begin{statement}{stm:47}
By (\ref{stm:45}), $h^{q+1} \in U$, so $H(h)$ is invariant.
\end{statement}

\begin{statement}{stm:48}
By (\ref{stm:47}), (\ref{stm:43}) and (\ref{stm:42}), $H(h) = G$, so $G$ is abelian.
\end{statement}