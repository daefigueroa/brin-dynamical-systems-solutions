\subsection*{Exercise 1.5.3.}

\begin{exercise}{stm:ex1.5.3-setup}
Suppose $p$ is an attracting fixed point for $f$. Show that there is a neighborhood $U$ of $p$ such that the forward orbit of every point in $U$ converges to $p$.
\end{exercise}

\subsection*{Proof {\color{blue} + reasoning}:}

\begin{explanation}{stm:ex1.5.3-m1}
Let's work out the meanings of the assumptions and of the required result.
\end{explanation}

\begin{explanation}{stm:ex1.5.3-q2}
What is an attracting fixed point for $f$? A fixed point $p$ is attracting if there exists a neighborhood $U$ of $p$ such that $\overline{U}$ is compact, $f(\overline{U}) \subseteq U$, and $\bigcap_{n \ge 0} f^n(U) = \{p\}$.
\end{explanation}

\begin{explanation}{stm:ex1.5.3-q4}
From the definition of $p$, we are already given a candidate neighborhood $U$ of $p$. Intuitively, it seems that $f^n(U)$ form a sequence of sets in which each is contained in the previous, with a single point $p$ in the intersection, so orbits should converge to $p$ or to a point in $\bigcap_{n \ge 0} f^n(\overline{U})$. But $f(\overline{U}) \subseteq U$ makes it likely that $\bigcap_{n \ge 0} f^n(\overline{U}) = \bigcap_{n \ge 0} f^n(U)$. Let's prove those intuitions.
\end{explanation}

\begin{explanation}{stm:ex1.5.3-q4a}
Is $f^{n+1}(U) \subseteq f^n(U)$ for all $n \in \mathbb{N}$?
\end{explanation}

\begin{statement}{stm:ex1.5.3-5}
By assumption, there exists a neighborhood $U$ of $p$ such that $\overline{U}$ is compact, $f(\overline{U}) \subseteq U$, and $\bigcap_{n \ge 0} f^n(\overline{U}) = \{p\}$.
\end{statement}


\begin{statement}{stm:ex1.5.3-6}
Clearly,
$$U \subset \overline{U}.$$
\end{statement}

\begin{statement}{stm:ex1.5.3-7}
From (\ref{stm:ex1.5.3-6}) and (\ref{stm:ex1.5.3-5})
$$f(U) \subseteq f(\overline{U}) \subseteq U.$$
\end{statement}

\begin{statement}{stm:ex1.5.3-10}
Therefore, 
$$f^{n+1}(U) \subseteq f^n(U) \text{ for all } n \in \mathbb{N}$$
\end{statement}
    
\begin{explanation}{stm:ex1.5.3-q11}
Is it true that $\bigcap_{n \ge 0} f^n(\overline{U}) = \bigcap_{n \ge 0} f^n(U)$?
\end{explanation}

\begin{statement}{stm:ex1.5.3-12}
Clearly,
$$\bigcap_{n \ge 0} f^n(U) \subseteq \bigcap_{n \ge 0} f^n(\overline{U})$$
\end{statement}

\begin{statement}{stm:ex1.5.3-13}
Conversely, 
$$\bigcap_{n \ge 0} f^n(\overline{U}) \subseteq \bigcap_{n \ge 1} f^n(\overline{U}) = \bigcap_{n \ge 0} f^{n+1}(\overline{U}) \subseteq \bigcap_{n \ge 0} f^n(U)$$
\end{statement}

\begin{statement}{stm:ex1.5.3-14}
So, from (\ref{stm:ex1.5.3-13})
$$\bigcap_{n \ge 0} f^n(U) = \bigcap_{n \ge 0} f^n(\overline{U})$$
\end{statement}

\begin{explanation}{stm:ex1.5.3-915}
The answer to (\ref{stm:ex1.5.3-q11}) is yes. Now, let's try to prove convergence of forward orbits. 
\end{explanation}

\begin{statement}{stm:ex1.5.3-916}
Let $x \in U$. Define $(x_n)_{n \in \mathbb{N}} = (f^n(x))_{n \in \mathbb{N}}$.
\end{statement}

\begin{explanation}{stm:ex1.5.3-q16}
Does $(x_n)$ converge to $p$?
\end{explanation}

\begin{explanation}{stm:ex1.5.3-a17}
Let's try proof by contradiction. 
\end{explanation}

\begin{statement}{stm:ex1.5.3-18}
Assume $(x_n)$ does not converge. Then $\exists \varepsilon' > 0$ such that $\forall n : \exists k \ge n : d(f^k(x), p) > \varepsilon'$.
\end{statement}

\begin{explanation}{stm:ex1.5.3-m15}
I'm stuck. I know there are infinitely many points in $U \setminus B(p, \varepsilon')$. Intuitively, this should contradict the fact that $\bigcap_{n \ge 0} f^n(\overline{U})$ only contains $p$.
\end{explanation}

\begin{explanation}{stm:ex1.5.3-m19}
Let me check what given assumptions I haven't used in my proof yet. I haven't used that $p$ is a fixed point, that $f$ is continuous, or the compactness of $\overline{U}$.
\end{explanation}

\begin{explanation}{stm:ex1.5.3-20}
Let me think of properties that we can use, or look up general consequences of these facts.
\end{explanation}

\begin{explanation}{stm:ex1.5.3-20b}
 In a metric space, the intersection of compact sets is compact.
\end{explanation}

\begin{explanation}{stm:ex1.5.3-22}
    If some $y \in U$ is not in $f^n(U)$ for some $n \in \mathbb{N}$, it won't be in any $f^k(U)$ for $k \ge n$.
\end{explanation}

\begin{explanation}{stm:ex1.5.3-21}
In a compact metric space, all sequences have convergent subsequences.
\end{explanation}

\begin{explanation}{stm:ex1.5.3-r23}
I think I can apply (\ref{stm:ex1.5.3-21}) to show a contradiction: 
\end{explanation}

\begin{statement}{stm:ex1.5.3-23}
From (\ref{stm:ex1.5.3-18}), there exists a sequence $f^{m_n}(x)$ such that $d(f^{m_n}(x), p) \ge \varepsilon$ for all $n \ge 0$. By compactness of $\overline{U}$, this sequence has a convergent subsequence $(f^{z_n}(x))_{n \ge 0}$ with $f^{z_n}(x) \to z \in \overline{U}$ and $z_n \to \infty$.
\end{statement}

\begin{statement}{stm:ex1.5.3-25a}
Since $f$ is continuous and $\overline{U}$ compact, $f^n(\overline{U})$ is compact for all $n \ge 0$. 
\end{statement}

\begin{statement}{stm:ex1.5.3-25}
$\forall n \ge 0$ there exists $K$ s.t. $f^{z_K}(x) \in f^n(\overline{U})$, hence $\forall m \ge K$, $f^{z_m}(x) \in f^n(\overline{U})$.
\end{statement}

\begin{statement}{stm:ex1.5.3-27}
From (\ref{stm:ex1.5.3-25}), the limit point $z$ must be in $f^n(\overline{U})$ for all $n \ge 0$.
\end{statement}

\begin{statement}{stm:ex1.5.3-28}
Therefore, $z \in \bigcap_{n \ge 0} f^n(\overline{U}) = \{p\}$.
\end{statement}

\begin{statement}{stm:ex1.5.3-29}
So, $z = p$, which contradicts (\ref{stm:ex1.5.3-18}), so assumption (\ref{stm:ex1.5.3-18}) is false. Hence $(x_n)$ converges.
\end{statement}

\begin{explanation}{stm:ex1.5.3-q31}
We still need to show that $(x_n)$ converges to $p$, but the proof will likely be almost the same as above.
\end{explanation}

\begin{statement}{stm:ex1.5.3-31}
Suppose $x_n \to q$ and $q \ne p$.
\end{statement}

\begin{statement}{stm:ex1.5.3-32}
Let $n \ge 0$. The sequence $(x_i)_{i \ge n}$ is contained in $f^n(\overline{U})$, which is compact, so $q \in f^n(\overline{U})$. Therefore, $q \in \bigcap_{n \ge 0} f^n(\overline{U})$.
\end{statement}

\begin{statement}{stm:ex1.5.3-34}
By (\ref{stm:ex1.5.3-14}), $q \in \bigcap_{n \ge 0} f^n(U) = \{p\}$, which is a contradiction. Hence, (\ref{stm:ex1.5.3-31}) is false.
\end{statement}

\begin{statement}{stm:ex1.5.3-36}
From (\ref{stm:ex1.5.3-34}) and (\ref{stm:ex1.5.3-29}) it follows that $x_n \to p$. Therefore, the forward orbit of any point in $U$ converges to $p$.
\end{statement}