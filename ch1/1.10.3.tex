\subsection*{Exercise 1.10.3.}

\begin{exercise}{stm:ex1.10.3-setup}
Let $f : \mathbb{R}^n \to \mathbb{R}$ be a smooth function. 
\end{exercise}

\begin{exercise}{stm:ex1.10.3-setup1}
Show that $-f$ is a Lyapunov function for the gradient flow.
\end{exercise}
    
\begin{exercise}{stm:ex1.10.3-setup2}
Show that the trajectories of the gradient flow are orthogonal to the level sets of $f$.
\end{exercise}

\subsection*{Proof {\color{blue} + reasoning}:}

\begin{explanation}{stm:ex1.10.3-3}
Let's write out the relevant definitions.
\end{explanation}

\begin{statement}{stm:ex1.10.3-4}
The gradient flow is the flow of the differential equation $\dot{x} = \nabla f(x)$.
\end{statement}

\begin{explanation}{stm:ex1.10.3-5}
Denote the time-$t$ gradient flow by $g^t : \mathbb{R}^n \to \mathbb{R}^n$. For all $x \in \mathbb{R}^n$ and $t \in \mathbb{R}^+$, write $g_x(t) := g^t(x)$.
\end{explanation}

\begin{explanation}{stm:ex1.10.3-6}
For each $x$ this defines $g_x : \mathbb{R}^+ \to \mathbb{R}^n$.
\end{explanation}

\begin{statement}{stm:ex1.10.3-7}
Let $x \in \mathbb{R}^n$ and $t \in \mathbb{R}^+$. Note $(f \circ g_x)(0) = f(x)$ and $(f \circ g_x)(t) = f(g^t(x))$.
\end{statement}

\begin{statement}{stm:ex1.10.3-8}
By (\ref{stm:ex1.10.3-7}), if $(f \circ g_x)'(s) \ge 0$ for all $s \in \mathbb{R}^+$ then $-f$ is Lyapunov.
\end{statement}

\begin{statement}{stm:ex1.10.3-10}
By (\ref{stm:ex1.10.3-4}), $g_x'(t) = \nabla f(g_x(t))$. 
\end{statement}

\begin{statement}{stm:ex1.10.3-11}
By the multivariate chain rule and (\ref{stm:ex1.10.3-10}), $$(f \circ g_x)'(t) = \langle \nabla f(g_x(t)), g_x'(t) \rangle = \langle g_x'(t), g_x'(t) \rangle$$ where $\langle \cdot, \cdot \rangle$ is the inner product in $\mathbb{R}^n$. 
\end{statement}

\begin{statement}{stm:ex1.10.3-12}
By definition of inner products, $\langle g_x'(t), g_x'(t) \rangle \ge 0$.
\end{statement}

\begin{statement}{stm:ex1.10.3-13}
By (\ref{stm:ex1.10.3-11}), (\ref{stm:ex1.10.3-12}) and (\ref{stm:ex1.10.3-8}), $-f$ is Lyapunov.
\end{statement}

\begin{explanation}{stm:ex1.10.3-14}
Next, we want to show statement (2).
\end{explanation}

\begin{explanation}{stm:ex1.10.3-15}
To express orthogonality, we need a common inner product space, but this is just $\mathbb{R}^n$ in our case.
\end{explanation}

\begin{explanation}{stm:ex1.10.3-insert15a}
Which vectors are we trying to prove are orthogonal? I think, given some point $x \in \mathbb{R}^n$, we should compare the time derivative of the orbit of $x$ at $t=0$, with a vector in $\mathbb{R}^n$ ‘tangent’ to the level set of $f$ at $f(x)$.
\end{explanation}

\begin{explanation}{stm:ex1.10.3-17}
How can we define the tangent vector?
\end{explanation}

\begin{statement}{stm:ex1.10.3-18}
Let $x \in \mathbb{R}^n$.
\end{statement}

\begin{statement}{stm:ex1.10.3-19}
Define the level set $C := f^{-1}(f(x))$.
\end{statement}

\begin{explanation}{stm:ex1.10.3-insert19a}
$C$ is a subset of $\mathbb{R}^n$, but I don't think it is necessarily a smooth manifold. Still, we can define tangent vectors in terms of smooth paths in $\mathbb{R}^n$:
\end{explanation}

\begin{statement}{stm:ex1.10.3-22}
Let $T_x = \{ \dot{\gamma}(0) : \exists \varepsilon > 0 \text{ s.t. } \gamma : (-\varepsilon, \varepsilon) \to C \text{ is smooth and } \gamma(0) = x \}$.
\end{statement}

\begin{statement}{stm:ex1.10.3-23}
By (\ref{stm:ex1.10.3-10}), $(f \circ g_x)'(0) = \nabla f(g_x(0)) = \nabla f(x) = g_x'(0)$.
\end{statement}

\begin{statement}{stm:ex1.10.3-24}
Let $V \in T_x$, with corresponding path $\gamma : (-\varepsilon, \varepsilon) \to C$.
\end{statement}

\begin{explanation}{stm:ex1.10.3-25}
We need to show $\langle V, \nabla f(x) \rangle = 0$.
\end{explanation}

\begin{statement}{stm:ex1.10.3-28}
Since $\gamma(t) \in C$ for all $t \in (-\varepsilon, \varepsilon)$, $f(\gamma(t)) = f(\gamma(0))$ for all $t \in (-\varepsilon, \varepsilon)$. 
\end{statement}

\begin{statement}{stm:ex1.10.3-29}
By the multivariate chain rule,
\begin{align*}
\langle \nabla f(x), V \rangle 
&= \sum_{k=1}^n V_k \frac{\partial f}{\partial y_k}(x) \\
&= \sum_{k=1}^n V_k \frac{\partial f}{\partial y_k}(\gamma(0)) \\
&= \sum_{k=1}^n \dot{\gamma}(0)_k \frac{\partial f}{\partial y_k}(\gamma(0)) \\
&= (f \circ \gamma)'(0).
\end{align*}
\end{statement}

\begin{statement}{stm:ex1.10.3-30}
By (\ref{stm:ex1.10.3-28}), $(f \circ \gamma)'(0) = 0$, so by (\ref{stm:ex1.10.3-29}), $\langle \nabla f(x), V \rangle = 0$, so by (\ref{stm:ex1.10.3-23}), the trajectories of the gradient flow are orthogonal to the level sets of $f$.
\end{statement}