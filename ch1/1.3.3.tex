\subsection*{Exercise 1.3.3}

\begin{exercise}{stm:ex1.3.3-setup133}
For $m \in \mathbb{Z}$, $|m| > 1$, define the times-$m$ map $E_m : S^1 \to S^1$ by $E_m x = mx \mod 1$. Show that the set of points with dense orbits is uncountable.
\end{exercise}

\subsection*{Proof {\color{blue} + reasoning}:}

\begin{explanation}{stm:ex1.3.3-m1}
My first idea is to show that for any irrational $x$, the orbit under $E_m$ is dense. How would we show this? We could use the semiconjugacy from $(\Sigma_m, \sigma)$ to $(S^1, E_m)$.
\end{explanation}

\begin{statement}{stm:ex1.3.3-m2}
As stated in ch1.3, the orbit of a point $0.x_1 x_2 \dots$ is dense in $S^1$ iff every finite sequence of elements in $\{0, \dots, m-1\}$ appears in the sequence $(x_i)_{i \in \mathbb{N}}$.
\end{statement}

\begin{explanation}{stm:ex1.3.3-p2}
Let's try proof by contradiction as described in (\ref{stm:ex1.3.3-m1}).
\end{explanation}

\begin{explanation}{stm:ex1.3.3-a3}
Let $x$ be an irrational number.
\end{explanation}

\begin{explanation}{stm:ex1.3.3-5}
$x$ has a base-$m$ expansion $0.x_1 x_2 \dots$
\end{explanation}

\begin{explanation}{stm:ex1.3.3-a6}
Suppose that the orbit of $x$ is not dense. There exists a finite sequence $a_1 \dots a_n$ of elements in $\{0, \dots, m-1\}$ that does not occur anywhere in $x_1 x_2 \dots$
\end{explanation}

\begin{explanation}{stm:ex1.3.3-q8}
Is the statement that any irrational $x$ has a dense orbit true? Let's try a different approach.
\end{explanation}

\begin{explanation}{stm:ex1.3.3-m13}
A more direct way to prove the statement is to construct an injective function from an uncountable set to the set of points in $S^1$ with dense orbits. It is noted in chapter 1.3 that we can construct a point in $S^1$ with dense orbit by simply concatenating all finite sequences. It seems likely that we can do something similar to construct the needed function.
\end{explanation}

\begin{statement}{stm:ex1.3.3-a13}
Let $U$ be the set of points in $S^1$ with a unique base-$m$ expansion. 
\end{statement}

\begin{statement}{stm:ex1.3.3-a13b}
By the remarks in section 1.3, $U$ is uncountable.
\end{statement}

\begin{statement}{stm:ex1.3.3-14}
Define $\phi : \Sigma_m \to S^1$. by $\phi((x_i)_{i \in \mathbb{N}}) := \sum_{i=1}^\infty x_i / m^i$
\end{statement}

\begin{statement}{stm:ex1.3.3-15}
By the remarks in section 1.3, $\phi$ is bijective on $\phi^{-1}(U)$.
\end{statement}{}

\begin{statement}{stm:ex1.3.3-a15}
Let $x \in U$, with base-$m$ expansion $(x_i)_{i \in \mathbb{N}}$.
\end{statement}

\begin{statement}{stm:ex1.3.3-a16}
Let $\mathcal{F}_m = \bigcup_{k=1}^\infty \{0, \dots, m-1\}^k$.
\end{statement}

\begin{statement}{stm:ex1.3.3-a16b}
Clearly, $\mathcal{F}_m$ is countable, so it can be indexed by $(\omega_i)_{i \in \mathbb{N}}$.
\end{statement}

\begin{statement}{stm:ex1.3.3-17}
Define $\alpha : U \to \Sigma_m$ by letting $\alpha(x) = x_1 \omega_1 x_2 \omega_2 x_3 \omega_3 \dots$, and define $\beta = \phi \circ \alpha$.
\end{statement}

\begin{statement}{stm:ex1.3.3-19}
Since every $y \in U$ has a unique base-$m$ expansion, $\alpha$ is injective, so by (\ref{stm:ex1.3.3-15}), $\beta$ is bijective. By construction, every finite sequence appears in $\alpha(y)$ for every $y \in U$, so by (\ref{stm:ex1.3.3-m2}), every point in $\beta(U)$ has a dense orbit.
\end{statement}

\begin{statement}{stm:ex1.3.3-20}
From (\ref{stm:ex1.3.3-20}), (\ref{stm:ex1.3.3-19}) and (\ref{stm:ex1.3.3-a13b}), we get that the set of all points in $S^1$ with dense orbits is uncountable.
\end{statement}
