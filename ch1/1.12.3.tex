\subsection*{Exercise 1.12.3.}

\begin{exercise}{stm:ex1.12.3-1123setup}
Compute the Lyapunov exponents for the solenoid.
\end{exercise}

\subsection*{Proof {\color{blue}+ reasoning}:}

\begin{explanation}{stm:ex1.12.3-1}
We need to calculate the matrix corresponding to the total derivative of $F^n$.
\end{explanation}

\begin{explanation}{stm:ex1.12.3-2}
Note the total derivative of $f \colon \mathbb{R}^n \to \mathbb{R}^n$ corresponds to the matrix
\[
\left[
\begin{array}{ccc}
\frac{\partial f_1}{\partial x_1} & \cdots & \frac{\partial f_1}{\partial x_n} \\
\vdots & \ddots & \vdots \\
\frac{\partial f_n}{\partial x_1} & \cdots & \frac{\partial f_n}{\partial x_n}
\end{array}
\right]
\]
\end{explanation}

\begin{explanation}{stm:ex1.12.3-3}
We need to compute $dF^n(x)v$. Is there a shortcut that we can take? Since we are calculating the derivative of a composition of functions, we might use the chain rule.
\end{explanation}

\begin{explanation}{stm:ex1.12.3-4}
The chain rule for total derivatives states $d(f \circ g)(a) = df(g(a)) \circ dg(a)$.
\end{explanation}

\begin{explanation}{stm:ex1.12.3-5}
In our case, $d(f^n(x)) = df(f^{n-1}(a)) \circ df^{n-1}(a) = \cdots$
\end{explanation}

\begin{explanation}{stm:ex1.12.3-6}
I'm not sure this will help us. Instead, let's calculate $f^n(x)$ directly, and from that the total derivative.
\end{explanation}

\begin{statement}{stm:ex1.12.3-7}
Let $F \colon S^1 \times D^2 \to S^1 \times D^2$ be the solenoid. Let $x, y \in \mathbb{R}$ and let $\lambda \in (0, \frac{1}{2})$.
\end{statement}

\begin{statement}{stm:ex1.12.3-8}
Note $F(\phi, x, y) = (2\phi, \lambda x + \frac{1}{2} \cos(2\pi\phi), \lambda y + \sin(2\pi\phi))$.
\end{statement}

\begin{statement}{stm:ex1.12.3-9}
By writing out the composition, we see that:
\begin{align*}
F^n(\phi, x, y)_1 &= 2^n \phi \\
F^n(\phi, x, y)_2 &= \lambda^n x + \frac{1}{2} \lambda^{n-1} \cos(2\pi \phi) + \cdots + \frac{1}{2} \lambda^0 \cos(2^{n-1} \pi \phi) \\
&= \lambda^n x + \frac{1}{2} \sum_{i=0}^{n-1} \lambda^i \cos(2^{n-1-i} \pi \phi) \\
F^n(\phi, x, y)_3 &= \lambda^n y + \frac{1}{2} \sum_{i=0}^{n-1} \lambda^i \sin(2^{n-1-i} \pi \phi)
\end{align*}
\end{statement}


\begin{statement}{stm:ex1.12.3-10}
By (\ref{stm:ex1.12.3-9}), denoting $\delta_{ij} := \frac{\partial F_i}{\partial z_j}(\phi, x, y)$, we can express $dF^n(\phi, x, y)$ as follows:
\begin{align*}
\delta_{11} &= 2^n \\
\delta_{21} &= -\frac{1}{2} \sum_{i=0}^{n-1} \lambda^i 2^{n-1-i} \pi \sin(2^{n-1-i} \pi \phi) \\
&= -\frac{\pi}{2} \sum_{i=0}^{n-1} \lambda^i 2^{n-1-i} \sin(2^{n-1-i} \pi \phi) \\
\delta_{31} &= \frac{\pi}{2} \sum_{i=0}^{n-1} \lambda^i 2^{n-1-i} \cos(2^{n-1-i} \pi \phi) \\
\delta_{22} &= \lambda^n \\
\delta_{33} &= \lambda^n \\
\delta_{ij} &= 0 \quad \text{otherwise}
\end{align*}
\end{statement}


\begin{statement}{stm:ex1.12.3-11}
The Lyapunov exponent is defined as
\[
\chi(\phi, x, y, v) = \lim_{n \to \infty} \frac{1}{n} \log \| dF^n(\phi, x, y)v \|
\]
\end{statement}

\begin{explanation}{stm:ex1.12.3-12}
Note the $\liminf$ in the above definition.
\end{explanation}

\begin{statement}{stm:ex1.12.3-14}
By (\ref{stm:ex1.12.3-10}),
\[
dF^n(\phi, x, y)v = v_1 (2^n + \delta_{21} + \delta_{31}) + \lambda^n (v_2 + v_3)
\]
\end{statement}

\begin{explanation}{stm:ex1.12.3-15}
Intuitively, if $v_1 = 0$, then the Lyapunov exponent seems to be $\log(\lambda)$, but if $v_1 \neq 0$, then the first term of $dF^n(\phi, x, y)v$ dominates and the exponent seems to be $\log(2)$, given that $\delta_{21}$ and $\delta_{31}$ (which depend on $n$ and $\phi$) are small enough.
\end{explanation}

\begin{explanation}{stm:ex1.12.3-16}
So, we need to argue about the limiting behaviour of these terms. My idea is to bound $\chi(\phi, x, y, v)$ from above and below. We can find an upper bound using the fact that (co)sines are bounded by 1 and the fact that $\lambda$ is less than $\frac{1}{2}$.
\end{explanation}

\begin{statement}{stm:ex1.12.3-17}
Note, since $\lambda \in (0, \frac{1}{2})$, that $|\delta_{21}| \le \pi \cdot n \cdot 2^n$ and $|\delta_{31}| \le \pi \cdot n \cdot 2^n$.
\end{statement}

\begin{statement}{stm:ex1.12.3-18}
Suppose $v_1 \neq 0$. By (\ref{stm:ex1.12.3-17}) and (\ref{stm:ex1.12.3-14}), for $n$ sufficiently large,
\begin{align*}
\frac{1}{n} \log \| dF^n(\phi, x, y)v \| 
&= \frac{1}{2} \cdot \frac{1}{n} \log \left( \| dF^n(\phi, x, y)v \|^2 \right) \\
&= \frac{1}{2n} \log \left( v^2 (2^n + \delta_{21} + \delta_{31})^2 + \lambda^{2n}(v_2 + v_3)^2 \right) \\
&\le \frac{1}{2n} \log \left( (v_1 \cdot 3 \pi \cdot n \cdot 2^n)^2 + \lambda^{2n}(v_2 + v_3)^2 \right) \\
&\le \frac{1}{2n} \log \left( (v_1 \cdot 4 \pi \cdot n \cdot 2^n)^2 \right) \\
&= \frac{1}{n} \log (v_1 \cdot 4 \pi \cdot n \cdot 2^n) \\
&= \frac{1}{n} \left( \log(v_1 \cdot 4 \pi \cdot n) + n \log(2) \right)\\
&\xrightarrow{n \to \infty} \log(2)
\end{align*}
\end{statement}

\begin{statement}{stm:ex1.12.3-19}
By (\ref{stm:ex1.12.3-18}), $\chi(\phi, x, y, v) \le \log(2)$.
\end{statement}

\begin{statement}{stm:ex1.12.3-20}
For the lower bound, by (\ref{stm:ex1.12.3-20}) and (\ref{stm:ex1.12.3-18}),
\begin{align*}
\frac{1}{n} \log \| dF^n(\phi, x, y)v \| 
&= \frac{1}{2n} \log \left( v_1^2(2^n + \delta_{21} + \delta_{31})^2 + \lambda^{2n}(v_2 + v_3)^2 \right) \\
&\ge \frac{1}{2n} \log \left( v_1^2 \cdot 2^{2n} \right) \\
&= \frac{1}{n} \log (v_1 \cdot 2^n) \\
&= \log(2) + \frac{1}{n} \log(v_1) \xrightarrow{n \to \infty} \log(2)
\end{align*}
\end{statement}


\begin{statement}{stm:ex1.12.3-21}
By (\ref{stm:ex1.12.3-19}) and (\ref{stm:ex1.12.3-20}), $\chi(\phi, x, y, v) = \log(2)$.
\end{statement}

\begin{statement}{stm:ex1.12.3-22}
Suppose $v_1 = 0$. By (\ref{stm:ex1.12.3-22}) and (\ref{stm:ex1.12.3-14}),
\begin{align*}
\frac{1}{n} \log \| dF^n(\phi, x, y)v \| 
&= \frac{1}{n} \log (\lambda^n (v_2 + v_3)) \\
&= \log(\lambda) + \frac{1}{n} \log(v_2 + v_3) \\
&\xrightarrow{n \to \infty} \log(\lambda)
\end{align*}
\end{statement}

\begin{statement}{stm:ex1.12.3-23}
By (\ref{stm:ex1.12.3-22}), $\chi(\phi, x, y, v) = \log(\lambda)$.
\end{statement}

\begin{statement}{stm:ex1.12.3-24}
By (\ref{stm:ex1.12.3-23}) and (\ref{stm:ex1.12.3-21}), the Lyapunov exponents are $\log(2)$ and $\log(\lambda)$. \qed
\end{statement}
