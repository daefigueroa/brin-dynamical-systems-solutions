\documentclass{article}
\usepackage{amsmath,amssymb,amsthm}
\usepackage{../proofsteps}

\begin{document}

\sectionlabel{Exercise 2.5.3.}

\begin{stm}{stm:253setup}[exercise]
Let $\{a_n\}$ be a subadditive sequence of non-negative real numbers, i.e.
\end{stm}

\begin{stm}{stm:253cond1}[exercise]
$0 \le a_{m+n} \le a_m + a_n$ for all $m,n \ge 0$.
\end{stm}

\begin{stm}{stm:253goal}[exercise]
Show that $\lim_{n \to \infty} \frac{a_n}{n} = \inf_{n \ge 0} \frac{a_n}{n}$.
\end{stm}

\sectionlabel{Proof {\color{blue}+ reasoning}:}

\begin{stm}{stm:reason1}[reasoning]
We need to show $\lim_{n \to \infty} \frac{a_n}{n}$ is the infimum, which we can do by checking that it is a lower bound of $\left\{ \frac{a_n}{n} \right\}$, and that it is greater than or equal to any lower bound of $\left\{ \frac{a_n}{n} \right\}$.
\end{stm}

\begin{stm}{stm:choosek}[proof]
Let $k \in \mathbb{N}_+$.
\end{stm}

\begin{stm}{stm:reason2}[reasoning]
We need to show $\lim_{n \to \infty} \frac{a_n}{n} \le \frac{a_k}{k}$. To do that, we could find $f(n) \to 0$ such that, for $n$ sufficiently large, 
\[
\frac{a_n}{n} \le \frac{a_k}{k} + f(n).
\]
What natural choice for $n$ can we make?
\end{stm}

\begin{stm}{stm:ngeqk}[proof]
Let $n \ge k$.
\end{stm}

\begin{stm}{stm:reason4}[reasoning]
To prove \ref{stm:reason2}, we will need to show that $\frac{a_n}{n} - \frac{a_k}{k} \to 0$ as $n \to \infty$. This should follow in some way from the nonnegativity and the subadditivity of $(a_i)$.
\end{stm}

\begin{stm}{stm:reason5}[reasoning]
Observe that for any natural number $m$, the value $a_{mk}$ is bounded by $m a_k$, therefore,

\[
\frac{a_{mk}}{mk} \le \frac{ma_k}{mk} = \frac{a_k}{k}.
\]
In other words, for the subsequence of $a_i$ where $i$ is a multiple of $k$ we have the required convergence. And, again by subadditivity, any other element is at most $k a_1$ away from this subsequence.
\end{stm}

\begin{stm}{stm:nform}[proof]
By \ref{stm:ngeqk}, $n = mk + m'$, where $m \in \mathbb{N}$ and $m' < k$.
\end{stm}

\begin{stm}{stm:applysubadd}[proof]
By \ref{stm:nform}, and the subadditivity of $(a_n)$,
\begin{align*}
\frac{a_n}{n} - \frac{a_k}{k} &=\frac{a_{mk + m'}}{n} - \frac{a_k}{k} \\
&\le \frac{a_{mk} + a_{m'}}{n} - \frac{a_k}{k} \\
&\le \frac{ma_k}{mk + m'} + \frac{k a_1}{n} - \frac{a_k}{k} \\
&\xrightarrow{n \to \infty} 0
\end{align*}
\end{stm}

\begin{stm}{stm:limlowerbound}[proof]
Hence, $\lim_{n \to \infty} \frac{a_n}{n}$ is a lower bound for $\left\{ \frac{a_n}{n} : n \ge 1 \right\}$.
\end{stm}

\begin{stm}{stm:infproof}[proof]
Additionally, if $C \le \frac{a_m}{m}$ for all $m \in \mathbb{N}$, then clearly $C \le \lim_{n \to \infty} \frac{a_n}{n}$, so by \ref{stm:limlowerbound},

\[
\lim_{n \to \infty} \frac{a_n}{n} = \inf_{n \ge 0} \frac{a_n}{n}.
\]
\end{stm}

\end{document}
