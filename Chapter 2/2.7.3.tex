\documentclass{article}
\usepackage{amsmath,amssymb,amsthm}
\usepackage{../proofsteps}

\begin{document}

\sectionlabel{Exercise 2.7.3.}

\begin{stm}{stm:273setup}[exercise]
Give a non-trivial example of a homeomorphism $f$ of a compact metric space $(X,d)$ such that $d(f^n(x), f^n(y)) \to 0$ as $n \to \infty$ for every pair $x,y \in X$.
\end{stm}

\sectionlabel{Proof {\color{blue}+ reasoning}:}

\begin{stm}{stm:insert1}[reasoning]
Let's recall simple examples of continuous maps of compact metric spaces, and see if they satisfy \ref{stm:273setup}.
\end{stm}

\begin{stm}{stm:insert2}[reasoning]
The translation $R_\alpha$ on $S^1$ is not valid, as it preserves distances.
\end{stm}

\begin{stm}{stm:insert3}[reasoning]
The map $h: x \mapsto \frac{1}{2}x$ on $S^1$ has the property that $d(h^n(x), h^n(y)) \to 0$ as $n \to \infty$ for every pair $x,y \in X$, but it is not surjective.
\end{stm}

\begin{stm}{stm:insert4}[reasoning]
To solve this issue, we can define a map that is the piecewise combination of a contraction (such as $f$) on one half of the circle and an expansion on the other half of the circle.
\end{stm}

\begin{stm}{stm:3}[proof]
Define $f: S^1 \to S^1$ by
\begin{align*}
f(x) = 
\begin{cases}
\frac{1}{2}x & \text{if } x \in [0, \frac{1}{2}) \\
\frac{3}{2}x - \frac{1}{2} & \text{if } x \in [\frac{1}{2}, 1) \\
\end{cases}
\end{align*}
\end{stm}

\begin{stm}{stm:4}[proof]
Clearly, $f$ is a homeomorphism such that $d(f^n(x), f^n(y)) \to 0$ as $n \to \infty$ for all pairs $x, y$.
\end{stm}

\end{document}