\documentclass{article}
\usepackage{amsmath,amssymb}
\usepackage{proofsteps}

\begin{document}

\sectionlabel{Exercise 1.1.2}

\begin{stm}{stm:statement}[exercise]
Suppose $(X,f)$ is a factor of $(Y,g)$ by a semi-conjugacy $\pi\colon Y \to X$.
\end{stm}

\begin{stm}{stm:setup}[exercise]
Show that if $y \in Y$ is a periodic point, then $\pi(y) \in X$ is periodic.
\end{stm}

\begin{stm}{stm:setup2}[exercise]
Give an example to show that the preimage of a periodic point does not necessarily contain a periodic point.
\end{stm}

\sectionlabel{Proof of exercise 1.1.2}

\begin{stm}{}[reasoning]
Let's start with \ref{stm:setup}.
\end{stm}

\begin{stm}{stm:verify}
Let $y \in Y$ be periodic.
\end{stm}

\begin{stm}{stm:write}[reasoning]
We can just write out the definitions and see if the correct answer follows.
\end{stm}

\begin{stm}{stm:calc}
$$
f(\pi(y)) \;=\; \pi\bigl(g(y)\bigr) \;=\; \pi(y).
$$
\end{stm}

\begin{stm}{stm:calc-conclusion}
So, \ref{stm:setup} follows from \ref{stm:calc}.
\end{stm}

\begin{stm}{stm:counterexample-intro}[reasoning]
Now, we want to construct a counterexample. Let's think of any semiconjugacy and iterate from there. An obvious example is a projection.
\end{stm}

\begin{stm}{stm:AB-sets}
Let $A$ and $B$ be sets, and $\pi_A\colon A \times B \to A$ the projection.
\end{stm}

\begin{stm}{stm:diagram}
For all $\alpha$ and $\beta$, the following diagram commutes:
\[
\begin{array}{c@{\quad}c@{\quad}c}
A \times B & \xrightarrow{\;\alpha \times \beta\;} & A \times B\\[6pt]
\downarrow{\pi_A} && \downarrow{\pi_A}\\[4pt]
A & \xrightarrow{\;\alpha\;} & A
\end{array}
\]
\end{stm}

\begin{stm}{stm:intuition}[reasoning]
Intuitively, by taking projections, we 'forget' about the effect of $\beta$. So, we can simply choose $\beta$ such that all points in $A \times B$ are non-periodic with respect to $\beta$, while choosing $\alpha$ so that all points in $A$ are periodic with respect to $\alpha$. 
\end{stm}

\begin{stm}{stm:final-example}[reasoning]
Let $A$ be any set, $B = [0,1]$, $\alpha = \mathrm{id}_A$ and $\beta: (x \mapsto \frac{1}{2}x)$
\end{stm}

\begin{stm}{}[reasoning]
Wait, $0$ is still a periodic point in the example above. Let's modify it.
\end{stm}

\begin{stm}{stm:final-example}
    Let $A$ be any set, $B = (0,1]$, $\alpha = \mathrm{id}_A$ and $\beta: (x \mapsto \frac{1}{2}x)$
\end{stm}
    
\begin{stm}{stm:counterexample}
Clearly, $\beta$ has no periodic points, so $\alpha \times \beta$ has no periodic points, but all points in $A$ are periodic with respect to $\alpha$. 
\end{stm}

\begin{stm}{}
By \ref{stm:counterexample}, \ref{stm:setup2} follows.
\end{stm}

\end{document}
