\subsection*{Exercise 3.8.3.}

\begin{exercise}{stm:ex3.8.3-setup}
A common coding scheme called modified frequency modulation (MFM) inserts a 0 between each two symbols unless they are both 0's, in which case it inserts a 1.
\end{exercise}

\begin{exercise}{stm:ex3.8.3-setup2}
Prove that the set of sequences produced by the MFM coding is a sofic system.
\end{exercise}

\subsection*{Proof {\color{blue}+ reasoning}:}

\begin{explanation}{stm:ex3.8.3-0}
The exercise implies that the set of sequences produced by the MFM is not a proper SFT, but only a factor of one.
\end{explanation}

\begin{explanation}{stm:ex3.8.3-1}
Clearly, $11$ and $0000$ are not allowed, but it doesn't seem that we can define the MFM as the complement of the sequences that contain these words.
\end{explanation}

\begin{explanation}{stm:ex3.8.3-2}
Let's use proposition 3.7.1 and find a finite directed labeled graph that represents the MFM.
\end{explanation}

\begin{explanation}{stm:ex3.8.3-3}
Note that in a finite directed labeled graph the edges are mapped to labels, but not necessarily injectively.
\end{explanation}

\begin{statement}{stm:ex3.8.3-10}
Consider the following graph.
\[
\begin{tikzpicture}[->,>=stealth,shorten >=1pt,auto,node distance=2.5cm,
                    semithick]
  \tikzstyle{every state}=[fill=gray!10,draw=black,text=black,scale=0.9]

  \node[state] (A)                    {};
  \node[state] (B) [right of=A]      {};
  \node[state] (C) [below of=B]      {};
  
  \path (A) edge [loop above] node {1} (A)
            edge [bend left]  node {0} (B)
        (B) edge [loop above] node {0} (B)
            edge [bend left]  node {1} (A)
            edge             node {0} (C)
        (C) edge [bend left] node {0} (A);
\end{tikzpicture}
\]
\end{statement}

\begin{statement}{stm:ex3.8.3-11}
This is a finite directed labeled graph that represents the set of sequences produced by the MFM, so by proposition 3.7.1, this system is sofic.
\end{statement}
