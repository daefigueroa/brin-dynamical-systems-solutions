\subsection*{Exercise 3.4.3.}

\begin{exercise}{stm:ex3.4.3-setup}
Let \( A = \begin{pmatrix} 1 & 1 \\ 1 & 0 \end{pmatrix} \). Calculate the zeta function of \( \Sigma^e_A \).
\end{exercise}

\subsection*{Proof {\color{blue}+ reasoning}:}

\begin{explanation}{stm:ex3.4.3-1}
To answer, we can probably skip the definition of the zeta function given in terms of fixed points, and use the expression from proposition 3.4.2.
\end{explanation}

\begin{statement}{stm:ex3.4.3-2}
By proposition 3.4.2, \(\zeta_A(z) = \left( \det(I - zA) \right)^{-1}\).
\end{statement}

\begin{statement}{stm:ex3.4.3-3}
Therefore, 
\[
\zeta_A(z) = \left( \det \begin{pmatrix} 1 - z & -z \\ -z & 1 \end{pmatrix} \right)^{-1} = \left( 1 - 2z - z^2 \right)^{-1}.
\]
\end{statement}
