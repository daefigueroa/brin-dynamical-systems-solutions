\sectionlabel{Exercise 3.1.3.}

\begin{stm}{stm:313setup}[exercise]
Use a higher block presentation to prove that for any block code $c: X \to Y$ there is a subshift $Z$ and an isomorphism $f: Z \to X$ such that $c \circ f: Z \to Y$ is a $(0,0)$-block code.
\end{stm}

\sectionlabel{Proof {\color{blue}+ reasoning}:}

\begin{stm}{stm:1}[proof]
Let $c: X \to Y$ be a block code, with corresponding function $\alpha: W_{a+b+1} \to \mathcal{A}_m$.
\end{stm}

\begin{stm}{stm:2}[reasoning]
Clearly, any higher block presentation $f$ of $X$ gives an isomorphism $f: X \to \operatorname{im}(f)$, but it is not clear which one will make $f \circ c$ a $(0,0)$-block code. Intuitively, the most natural choice is the presentation of $X$ in which the blocks are aligned with those given by $c$. Let's check if this presentation has the required properties.
\end{stm}

\begin{stm}{stm:3}[proof]
Letting $k = a + b + 1$ and $l = b$, the higher block presentation $d$ of $X$ can be written as
\begin{align*}
d(x)_i = x_{i-a} \ldots x_{i+b}, \quad i \in \mathbb{Z}
\end{align*}
\end{stm}

\begin{stm}{stm:4}[proof]
Since $\operatorname{im}(d) \subseteq \Sigma_{W_{a+b+1}(X)}$ we have $W_1(\operatorname{im}(d)) \subseteq W_{{a+b+1}}(X)$.
\end{stm}

\begin{stm}{stm:5}[proof]
If $\omega \in W_{a+b+1}(X)$, then for some sequence $x \in X$ and $i \in \mathbb{Z}$,
\[
\omega = x_{i-a} \ldots x_{i+b}, \quad \text{so} \quad d(x)_i = \omega,
\]
so $\omega \in W_1(\operatorname{im}(d))$.
\end{stm}

\begin{stm}{stm:5}[proof]
By \ref{stm:5} and \ref{stm:4}, $W_1(\operatorname{im}(d)) = W_{a+b+1}(X)$.
\end{stm}

\begin{stm}{stm:6}[proof]
By Exercise 3.1.2, $d$ is an isomorphism onto its image. Let $d^{-1}: \operatorname{im}(d) \to X$ be its inverse.
\end{stm}

\begin{stm}{stm:7}[proof]
If $z \in \operatorname{im}(d)$ and $i \in \mathbb{Z}$, then there exists a unique $x$ such that $d(x) = z$, so
\begin{align*}
(c \circ d^{-1})(z)_i &= c(d^{-1}(z))_i \\
&= c(x)_i \\
&= \alpha(x_{i-a} \ldots x_{i+b}) \\
&= \alpha(z_i)
\end{align*}
\end{stm}

\begin{stm}{stm:8}[proof]
By \ref{stm:7} and \ref{stm:5}, $c \circ d^{-1}: \operatorname{im}(d) \to Y$ is a $(0,0)$-block code and, by \ref{stm:6}, $d^{-1}$ is an isomorphism.
\end{stm}
