\documentclass{article}
\usepackage{amsmath,amssymb}
\usepackage{../proofsteps}

\begin{document}

\sectionlabel{Exercise 1.4.3.}

\begin{stm}{setup}[exercise]
Verify that the metrics on $\Sigma_m$ and $\Sigma_m^+$ generate the product topology
\end{stm}

\sectionlabel{Proof {\color{blue} + reasoning}:}

\begin{stm}{stm:q2}[reasoning]
Firstly, what product topology is being referred to? $\Sigma_m = A_m^{\mathbb{Z}}$, which is the space of sequences of elements in $\{1,\ldots,m\}$ indexed by $\mathbb{Z}$, which can be seen as a product $\prod_{z \in \mathbb{Z}} A_m$.
\end{stm}

\begin{stm}{stm:3}[reasoning]
The product topology is generated by products of open sets. Is there a general result that countable product topologies are generated by cylinders, instead of just by countable products of open sets?
\end{stm}

\begin{stm}{stm:q4}[reasoning]
Yes, the result is that the $n$-dimensional cylinders form a basis for the product topology. This is exactly the topology on $\Sigma_m$ and $\Sigma_m^+$.
\end{stm}

\begin{stm}{stm:ga1}[reasoning]
How does \ref{stm:q4} help us? From a metric we can generate the collection of open balls around all points. A metric generates a topology when, given a basis set $B$ and any point $x \in B$, there is an open ball containing $x$ that is contained in $B$.
\end{stm}

\begin{stm}{stm:a5}[final-proof]
Let $C := C_{j_1,\ldots,j_k}^{n_1,\ldots,n_k} = \{ x = (x_\ell) : x_{n_i} = j_i, i=1,\ldots,k \}$ where $n_1 < n_2 < \cdots < n_k$ are indices in $\mathbb{Z}$ or $\mathbb{N}$, and $j_i \in A_m$.
\end{stm}

\begin{stm}{stm:6}[final-proof]
    Let $x := (x_i)  \in C$.
\end{stm}

\begin{stm}{stm:l6}[reasoning]
We want to show that $C$ contains an open ball $B$ such that $x \in B$.
\end{stm}

\begin{stm}{stm:a7}[reasoning]
Recall that the metrics on $\Sigma_m$ and $\Sigma_m^+$ are given by $d(x,x') = 2^{-l}$, where $l = \min \{ |i| : x_i \ne x_i' \}$. If we pick $\varepsilon$ small enough, we can construct $B(x, \varepsilon)$ such that all points in $B(x, \varepsilon)$ agree with $x$ up to the ‘largest’ index in the definition of $C$.
\end{stm}

\begin{stm}{stm:8a}[final-proof]
Let $m = \max \{ |n_i| : i \leq k \}$,$y \in B(x, 2^{-m})$, and $l = \min \{ |i| : y_i \ne x_i \}$. 
\end{stm}

\begin{stm}{stm:8}[final-proof]
We have $2^{-l} = d(x,y) < 2^{-m}$.
\end{stm}

\begin{stm}{stm:9}[final-proof]
From \ref{stm:8}, $l > m$, so $x_{n_i} = y_{n_i} \ \forall i \leq k$, hence $y \in C$.
\end{stm}

\begin{stm}{stm:11}[final-proof]
Therefore $B(x, 2^{-m}) \subseteq C$.
\end{stm}

\begin{stm}{stm:12}[final-proof]
By \ref{stm:11}, and the fact that the collection of sets such as $C_{j_1,\ldots,j_k}^{n_1,\ldots,n_k}$ form a basis for the product topology, the metrics generate the product topology.
\end{stm}

\end{document}
